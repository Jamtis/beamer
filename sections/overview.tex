% !TEX root = ../paper.tex
\section{Technical Overview}
\label{sec:overview}

\begin{sitemize}
    \item \(\Gen\parr{1^\secpar} \mapsto \parr{\pk,\sk}\):
    the public and secret key are defined as \(\pk \coloneqq \parr{N, g_0, ..., g_d, X}\) and \(\sk \coloneqq \parr{N, g_0, ..., g_d, \alpha}\).
    
    \item \(\Enc\parr{\pk} \mapsto \parr{C,K}\):
    sample \(r \gets \pars{\parr{N-1}/4}\),
    set \(R \coloneqq g_0^{r z}\),
    set \(T \coloneqq g_0^{r 2^{\ell_{\mathsf{T}}}}\),
    set \(K \coloneqq \operatorname{BBS}_N\parr{T}\),
    set \(t \coloneqq \operatorname{T}\parr{R}\),
    set \(S \coloneqq \abs{\parr{\prod_{i=0}^d g_i^{t^i} X}^r}\),
    and set \(C \coloneqq \parr{R,S}\).
    \item \(\Punct\parr{\sk,C'_1,...,C'_d} \mapsto \sk'\):
    set \(v_i \coloneqq \operatorname{dlog}\parr{g_i}\),
    let \(V\parr{Z} \coloneqq \sum_{i=0}^d v_i Z^i\),
    set \(\delta \coloneqq \sum_{i=1}^d V\parr{t'_i}\),
    set \(\beta \coloneqq \alpha - \delta/z\),
    set \(\sk' \coloneqq \parr{\beta, \delta}\).
\end{sitemize}


\begin{theorem}
    Assuming indistinguishability obfuscation,
    puncturable PRF,
    statistically binding commitment,
    and pseudorandom generator,
    there exists a deterministic statistically sound NIWI in the CRS model.
\end{theorem}

\begin{proof}
    Let \(\iO\) be an indistinguishability obfuscator,
    let \(\PPRF = \parr{\KeyGen, \Eval, \Punct}\) be a 2-puncturable,
    let \(\parr{\Com,\Open}\) be a statistically binding commitment scheme,
    and let \(\PRG\) be a pseudorandom generator.
    \\
    We define the following \(\NIWI \coloneqq \parr{\Setup, \Prove, \Vfy}\):
    \begin{sitemize}
        \item \(\Setup\parr{1^\secpar}\):
        Sample \(k_1,k_2 \gets \KeyGen\parr{1^\secpar}\).
        \\
        Let \(\mathsf{P} = \iO\parr{\parr{x, w} \mapsto \parr{\pi_0,\pi_1}}\) where \(\pi_0 \coloneqq \PRG\parr{\Eval\parr{k_1,x||w}}\) and \(\pi_1 \coloneqq \Com\parr{w; \Eval\parr{k_2,\pi_0}}\).
        \\
        Let \(\mathsf{V} = \iO\parr{\parr{x, \pi} \mapsto R\parr{x,w} = 1 \wedge \pi_0 = \PRG\parr{\Eval\parr{k_1,x||w}}}\) where \(w \coloneqq \Open\parr{\pi_1, \Eval\parr{k_2,\pi_0}}\).

        \item \(\Prove\parr{\CRS,x,w} \coloneqq \mathsf{P}\parr{x,w}\)

        \item \(\Vfy\parr{\CRS,x,\pi} \coloneqq \mathsf{V}\parr{x,\pi}\)
    \end{sitemize}

    Correctness directly follows from the correctness of the underlying primitives.
    Since the verification algorithm checks \(R\parr{x,w}\) the NIWI is perfectly (statistically) sound if the obfuscator is perfectly (statistically) correct.
    Now, we show the witness indistinguishability of \(\NIWI\) by a sequence of hybrid games.
    For simplicity we assume a perfectly binding commitment scheme.%
    \footnote{If the commitment scheme is statistically binding, then the saem argumentation applies but one has to consider the event that the scheme is not perfectly binding.}
    Let \(L \in \NP\) be language with relation \(R\).
    Let \(\hat{x} \in L\) be a statement with witnesses \(\hat{w}^0,\hat{w}^1\) s.t.\ \(\parr{\hat{x},\hat{w}^0},\parr{\hat{x},\hat{w}^1} \in R\).
    Let
    \begin{bralign}
        \mathsf{P}_{\hat{\pi}^0,\hat{\pi}^1,k_1,k_2}
        \coloneqq
        \iO\parr{\parr{x,w} \mapsto \begin{cases}
            \hat{\pi}^0 & x||w = \hat{x}||\hat{w}^0
            \\
            \hat{\pi}^1 & x||w = \hat{x}||\hat{w}^1
            \\
            \pi_0 \coloneqq \PRG\parr{\Eval\parr{k_1,x||w}}, \pi_1 \coloneqq \Com\parr{w; \Eval\parr{k_2, \pi_0}} & \text{else}\end{cases}
        }
    \end{bralign}
    and let
    \begin{bralign}
        \mathsf{V}_{\hat{\pi}^0,\hat{\pi}^1,k_1,k_2}
        \coloneqq
        \iO\parr{\parr{x,\pi} \mapsto \begin{cases}
            1 & \pi = \hat{\pi}^0
            \\
            1 & \pi = \hat{\pi}^1
            \\
            R\parr{x,w} = 1 \wedge \pi_0 = \PRG\parr{\Eval\parr{k'_1,x||w}} & \text{else}
        \end{cases}}
    \end{bralign}
    where \(w \coloneqq \Open\parr{\pi_1, \Eval\parr{k_2,\pi_0}}\).
    \begin{hybrids}
        \item Sample \(\CRS \gets \Setup\parr{1^\secpar}\).
        Let \(\pi^* \coloneqq \mathsf{P}\parr{\hat{x}||\hat{w}^0}\).
        Output \(\parr{\CRS, \pi^*}\).
        Note that this hybrid corresponds to the honest proving only uses the witness \(\hat{w}^0\).

        \item Let \(k'_1 \coloneqq \Punct\parr{k_1, \hat{x}||\hat{w}^0, \hat{x}||\hat{w}^1}\),
        let \(r_0^b \coloneqq \Eval\parr{k_1,\hat{x}||\hat{w}^b}\),
        let \(\hat{\pi}_0^b \coloneqq \PRG\parr{r_0^b}\),
        let \(r_1^b \coloneqq \Eval\parr{k_2,\hat{\pi}_0^b}\),
        and let \(\hat{\pi}_1^b \coloneqq \Com\parr{\hat{w}^b; r_1^b}\) for both \(b \in \bit\).
        Let \(\CRS \coloneqq \parr{\mathsf{P}_{\hat{\pi}^0,\hat{\pi}^1,k'_1,k_2}, \mathsf{V}_{\hat{\pi}^0,\hat{\pi}^1,k'_1,k_2}}\).

        \item Sample \(r_0^b \gets \bit^\secpar\).

        \item Sample \(\hat{\pi}_0^b \gets \bit^{2\secpar}\).
        Denote by \(E\) the event that \(\hat{\pi}_0^0, \hat{\pi}_0^1 \not\in \PRG\parr{\bit^\secpar}\).

        \item Let \(k'_2 \coloneqq \Punct\parr{k_2, \hat{\pi}_0^0, \hat{\pi}_0^1}\).
        Let \(\CRS \coloneqq \parr{\mathsf{P}_{\hat{\pi}^0,\hat{\pi}^1,k'_1,k'_2}, \mathsf{V}_{\hat{\pi}^0,\hat{\pi}^1,k'_1,k'_2}}\).

        \item Sample \(r_1^b \gets \bit^\secpar\).

        \item Let \(\hat{\pi}_1^b \coloneqq \Com\parr{\hat{w}^{1-b}; r_1^b}\).
        Let \(\pi^* \coloneqq \mathsf{P}\parr{\hat{x}||\hat{w}^1}\).

        \item Sample \(r_1^b \coloneqq \Eval\parr{k_2,\hat{\pi}_0^b}\).

        \item Let \(\CRS \coloneqq \parr{\mathsf{P}_{\hat{\pi}^0,\hat{\pi}^1,k'_1,k_2}, \mathsf{V}_{\hat{\pi}^0,\hat{\pi}^1,k'_1,k_2}}\).

        \item Let \(\hat{\pi}_0^b \coloneqq \PRG\parr{r_0^b}\).
        
        \item Let \(r_0^b \coloneqq \Eval\parr{k_1,\hat{x}||\hat{w}^b}\).
        Let \(\pi^* \coloneqq \mathsf{P}\parr{\hat{x}||\hat{w}^1}\).
        Output \(\parr{\CRS, \pi^*}\).
        Note that this hybrid corresponds to the honest proving only uses the witness \(\hat{w}^1\).
    \end{hybrids}
    For any PPT distinguisher \(\D\) consider the probability \(\varepsilon_i \coloneqq \Pr[{\parr{\CRS,\pi^*} \gets \H_i}]{\D\parr{\CRS,\pi^*} = 0}\) that the distinguisher outputs \(0\) in hybrid \(i\).
    \\
    Because the circuits \(\mathsf{P}\) and \(\mathsf{P}_{\hat{\pi}^0,\hat{\pi}^1,k'_1,k_2}\), and \(\mathsf{V}\) and \(\mathsf{V}_{\hat{\pi}^0,\hat{\pi}^1,k'_1,k_2}\) are functionally equivalent it follows that \(\abs{\varepsilon_1 - \varepsilon_2} \leq \eta_{\iO}\parr{\secpar}\).
    Because \(\H_2\) and \(\H_3\) only use \(k_1\) to compute \(r_0^b\) it follows that \(\abs{\varepsilon_2 - \varepsilon_3} \leq \eta_{\PPRF}\parr{\secpar}\) by the pseudorandomness of the puncturable PRF.
    Because \(\H_3\) and \(\H_4\) use \(r_0^b\) only to compute \(\pi_0^b\) it follows that \(\abs{\varepsilon_4 - \varepsilon_4} \leq \eta_{\PRG}\parr{\secpar}\) by the pseudorandomness of the PRG.
    Note
    \begin{bralign}
        \varepsilon_4
        &=
        \underbrace{\Pr[{\parr{\CRS,\pi^*} \gets \H_4}]{\D\parr{\CRS,\pi^*} = 0 \given E}}_{\varepsilon'_4} \underbrace{\Pr[{\parr{\CRS,\pi^*} \gets \H_4}]{E}}_{\leq 1}
        \\
        &+
        \underbrace{\Pr[{\parr{\CRS,\pi^*} \gets \H_4}]{\D\parr{\CRS,\pi^*} = 0 \given \text{not } E}}_{\leq 1} \underbrace{\Pr[{\parr{\CRS,\pi^*} \gets \H_4}]{\text{not } E}}_{\leq 2^{1-\secpar}}
        \\
        &\leq
        2^{1-\secpar}
        +
        \varepsilon'_4
    \end{bralign}
    and
    \begin{bralign}
        \varepsilon_4
        &=
        \varepsilon'_4 \underbrace{\Pr[{\parr{\CRS,\pi^*} \gets \H_4}]{E}}_{\geq 1-2^{1-\secpar}}
        +
        \Pr[{\parr{\CRS,\pi^*} \gets \H_4}]{\D\parr{\CRS,\pi^*} = 0 \given \text{not } E} \underbrace{\Pr[{\parr{\CRS,\pi^*} \gets \H_4}]{\text{not } E}}_{\geq 0}
        \\
        &\geq
        \varepsilon'_4 \parr{1-2^{1-\secpar}}
        \ ,
    \end{bralign}
    hence \(\abs{\varepsilon_4 - \varepsilon'_4} \leq 2^{1-\secpar}\).
    The event \(E\) implies that the circuits \(\mathsf{P}_{\hat{\pi}^0,\hat{\pi}^1,k'_1,k_2}\) and \(\mathsf{P}_{\hat{\pi}^0,\hat{\pi}^1,k'_1,k'_2}\) are functionally equivalent because \(\Eval\) only gets called on PRG images and \(k'_2\) in punctured at a non-PRG image.
    Moreover, \(E\) implies that the circuits \(\mathsf{V}_{\hat{\pi}^0,\hat{\pi}^1,k'_1,k_2}\) and \(\mathsf{V}_{\hat{\pi}^0,\hat{\pi}^1,k'_1,k'_2}\) are functionally equivalent.
    Specifically, on input \(\hat{x}||\hat{\pi}^b\) both circuits outputs \(1\).
    On the other hand for any input \(\hat{x}||\widetilde{\pi}_0^b||\hat{\pi}_1^b\) with \(\widetilde{\pi}_0^b \neq \hat{\pi}_0^b\) the outputs is \(0\) because the recovered \(w \coloneqq \Open\parr{\hat{\pi}_1^b, \Eval\parr{k'_2,\widetilde{\pi}_0^b}} = \bot\) (this follows from the prefect binding property of the commitment) will never satisfy the relation \(R\).
    Consequently, \(\abs{\varepsilon'_4 - \varepsilon'_5} \leq \eta_{\iO}\parr{\secpar}\) by the indistinguishability of the obfuscations.
    \\
    In \(\H_5\) the key \(k_2\) is only used to compute \(r_1^b\) and derive the key \(k'_2\) it follows \(\abs{\varepsilon'_5 - \varepsilon'_6} \leq \eta_{\PPRF}\parr{\secpar}\) by the pseudorandomness of the puncturable PRF.
\end{proof}


\subsection{Partially-Puncturable PKE}

\begin{definition}[Partially-Puncturable PKE]
    A partially-puncturable PKE (ppPKE) \(\PKE \coloneqq \parr{\Gen, \Enc_0, \Enc_1, \Dec, \Punct}\) fulfills the following properties:
    \begin{sitemize}
        \item \(\Gen\parr{1^\secpar}\) outputs a key pair \(\parr{\pk,\sk} \gets \Gen\parr{1^\secpar}\).
        \item \(\Enc\parr{\pk,m;r}\) outputs a two-part ciphertext \(c \coloneqq \parr{c_0,c_1}\).
        \item \(\Dec\parr{\sk,c}\) decrypts the ciphertext \(c\).
        \item \(\Punct\parr{\sk,c_0}\) outputs a secret-key \(\sk\pars{c_0}\) that is punctured on the first part of a ciphertext \(c_0\) s.t.\ \(\sk\pars{c_0}\) is unable to decrypt any ciphertext that has \(c_0\) as the first part.
        \item Correctness:
        \(\forall \parr{\pk,\sk} \in \Gen\ \forall m,r : \Dec\parr{\sk,\parr{c_0,c_1}}\) where \(c_0 \coloneqq \Enc_0\parr{\pk;r}\) and \(c_1 \coloneqq \Enc_1\parr{\pk,m;r}\).
        \item Correctness of punctured secret-key:
        \(\forall \parr{\pk,\sk} \in \Gen\ \forall c'_0,m,r : c_0 \neq c'_0 \implies \Dec\parr{\sk\pars{c'_0},\parr{c_0,c_1}} = m\) where \(c_0 \coloneqq \Enc_0\parr{\pk;r}\) and \(c_1 \coloneqq \Enc_1\parr{\pk,m;r}\) and \(\sk\pars{c'_0} \gets \Punct\parr{\sk,c'_0}\).
    \end{sitemize}
\end{definition}

\begin{definition}[IND-pKL security]
    A ppPKE is IND-pKL secure iff any PPT distinguisher \(\D = \parr{\D_1,\D_2}\) has negligible advantage, i.e., \(\abs{\Pr{\mathsf{EXP}^0_\D\parr{\secpar} = 0} - \Pr{\mathsf{EXP}^1_\D\parr{\secpar} = 0}} \leq \negl\parr{\secpar}\) where \(\mathsf{EXP}^b_\D\) is the following experiment:
    \begin{sitemize}
        \item sample \(\parr{\pk,\sk} \gets \Gen\parr{1^\secpar}\),
        \item sample \(r^* \gets \bit^\secpar\),
        \item sample \(b \gets \bit\),
        \item set \(\parr{m_0,m_1,\sigma} \gets \D_1\parr{1^\secpar,\pk}\),
        \item set \(c_0^* \coloneqq \Enc_0\parr{\pk;r^*}\),
        \item set \(\sk\pars{c^*_0} \coloneqq \Punct\parr{\sk,c_0^*}\),
        \item set \(c_1^* \coloneqq \Enc_1\parr{\pk,m_b;r^*}\),
        \item set \(b' \gets \D_2\parr{\sigma,\parr{c^*_0,c^*_1},\sk\pars{c^*_0}}\),
        \item output \(b = b'\).
    \end{sitemize}
\end{definition}


\subsection{pKDM-pKL security}

\begin{definition}[pKDM-pKL security]
    A ppPKE is pKDM-pKL secure iff any PPT distinguisher \(\D\) has negligible advantage, i.e., \(\abs{\Pr{\mathsf{EXP}^0_\D\parr{\secpar} = 0} - \Pr{\mathsf{EXP}^1_\D\parr{\secpar} = 0}} \leq \negl\parr{\secpar}\) where \(\mathsf{EXP}^b_\D\) is the following experiment:
    \begin{sitemize}
        \item sample \(\parr{\pk,\sk} \gets \Gen\parr{1^\secpar}\),
        \item sample \(r^* \gets \bit^\secpar\),
        \item sample \(b \gets \bit\),
        \item set \(c_0^* \coloneqq \Enc_0\parr{\pk;r^*}\),
        \item set \(\sk\pars{c^*_0} \coloneqq \Punct\parr{\sk,c_0^*}\),
        \item set \(m_0 \coloneqq 0\) and \(m_1 \coloneqq \sk\pars{c^*_0}\),
        \item set \(c_1^* \coloneqq \Enc_1\parr{\pk,m_b;r^*}\),
        \item set \(b' \gets \D\parr{1^\secpar,\pk,\parr{c^*_0,c^*_1},\sk\pars{c^*_0}}\),
        \item output \(b = b'\).
    \end{sitemize}
\end{definition}

\begin{lemma}
    Let \(\PKE \coloneqq \parr{\Gen, \Enc, \Dec}\) be a IND-CPA secure PKE scheme.
    Let \(\PKE' \coloneqq \parr{\Gen', \Enc'_0, \Enc'_1, \Dec', \Punct'}\) be a ppPKE.
    Then there exists a pKDM-pKL secure ppPKE \(\overline{\PKE} \coloneqq \parr{\overline{\Gen}, \overline{\Enc}_0, \overline{\Enc}_1, \overline{\Dec}, \overline{\Punct}}\).
\end{lemma}

\begin{proof}
    We first give a construction and then prove its security.
    \paragraph{Construction.}
    \begin{sitemize}
        \item \(\overline{\Gen}\parr{1^\secpar;r}\):
        output \(\parr{\overline{\pk},\overline{\sk}} \coloneqq \parr{\pk',\sk'} = \Gen'\parr{1^\secpar;r}\).

        \item \(\overline{\Enc}_0\parr{\overline{\pk};r}\):
        parse \(\parr{r_0,r_1,r_{\Gen}} \coloneqq r\),
        set \(\parr{\pk,\sk} \coloneqq \Gen\parr{1^\secpar;r_{\Gen}}\),
        output \(\overline{c}_0 \coloneqq \Enc'_0\parr{\pk';r_0}\).

        \item \(\overline{\Enc}_1\parr{\overline{\pk},m;r}\):
        parse \(\parr{r_0,r_1,r_{\Gen}} \coloneqq r\),
        set \(\parr{\pk,\sk} \gets \Gen\parr{1^\secpar;r_{\Gen}}\),
        set \(c'_1 \coloneqq \Enc'_1\parr{\pk',\sk;r_0}\),
        set \(c \coloneqq \Enc\parr{\pk,m;r_1}\),
        output \(\overline{c}_1 \coloneqq \parr{c'_1,c}\).

        \item \(\overline{\Dec}\parr{\overline{\sk},\overline{c}}\):
        parse \(\parr{c'_0,c'_1,c} \coloneqq \overline{c}\),
        decrypt \(\sk \coloneqq \Dec'\parr{\sk',\parr{c'_0,c'_1}}\),
        output \(m \coloneqq \Dec\parr{\sk,c}\).

        \item \(\overline{\Punct}\parr{\overline{\sk},\overline{c}_0}\):
        output \(\sk\pars{\overline{c}_0} \coloneqq \Punct'\parr{\sk',\overline{c}_0}\).
    \end{sitemize}

    \paragraph{Security proof.}
    We prove security through four hybrid games.
    \begin{hybrids}
        \item Sample \(\parr{\overline{\pk},\overline{\sk}} \gets \overline{\Gen}\parr{1^\secpar}\),
        sample \(r \gets \bit^\secpar\),
        set \(\overline{c}^*_0 \coloneqq \overline{\Enc}_0\parr{\pk;r}\),
        set \(\sk\pars{c^*_0} \coloneqq \Punct\parr{\sk,c^*_0}\),
        set \(m_0 \coloneqq 0\) and \(m_1 \coloneqq \sk\pars{c^*_0}\),
        sample \(b \gets \bit\),
        set \(\overline{c}^*_1 \coloneqq \overline{\Enc}_1\parr{\pk,m_b;r}\),
        output \(\parr{\overline{\pk},\overline{c}^*,\sk\pars{c^*_0}}\).

        \item Sample \(\parr{\overline{\pk},\overline{\sk}} \gets \overline{\Gen}\parr{1^\secpar}\),
        sample \(r = \parr{r_0,r_1,r_{\Gen}} \gets \bit^\secpar\),
        set \(\overline{c}^*_0 \coloneqq \overline{\Enc}_0\parr{\pk;r}\),
        set \(\sk\pars{c^*_0} \coloneqq \Punct\parr{\sk,c^*_0}\),
        set \(m_0 \coloneqq 0\) and \(m_1 \coloneqq \sk\pars{c^*_0}\),
        sample \(b \gets \bit\),
        set \(c'_1 \coloneqq \Enc'_1\parr{\pk',\fbox{\sk};r_0}\),
        set \(c \coloneqq \Enc\parr{\pk,m_b;r_1}\),
        set \(\overline{c}^* \coloneqq \parr{\overline{c}^*_0, c'_1, c}\),
        output \(\parr{\overline{\pk},\overline{c}^*,\sk\pars{c^*_0}}\).

        \item Sample \(\parr{\overline{\pk},\overline{\sk}} \gets \overline{\Gen}\parr{1^\secpar}\),
        sample \(r = \parr{r_0,r_1,r_{\Gen}} \gets \bit^\secpar\),
        set \(\overline{c}^*_0 \coloneqq \overline{\Enc}_0\parr{\pk;r}\),
        set \(\sk\pars{c^*_0} \coloneqq \Punct\parr{\sk,c^*_0}\),
        set \(m_0 \coloneqq 0\) and \(m_1 \coloneqq \sk\pars{c^*_0}\),
        sample \(b \gets \bit\),
        set \(c'_1 \coloneqq \Enc'_1\parr{\pk',\fbox{0};r_0}\),
        set \(c \coloneqq \Enc\parr{\pk,\fbox{\(m_b\)};r_1}\),
        set \(\overline{c}^* \coloneqq \parr{\overline{c}^*_0, c'_1, c}\),
        output \(\parr{\overline{\pk},\overline{c}^*,\sk\pars{c^*_0}}\).

        \item Sample \(\parr{\overline{\pk},\overline{\sk}} \gets \overline{\Gen}\parr{1^\secpar}\),
        sample \(r = \parr{r_0,r_1,r_{\Gen}} \gets \bit^\secpar\),
        set \(\overline{c}^*_0 \coloneqq \overline{\Enc}_0\parr{\pk;r}\),
        set \(\sk\pars{c^*_0} \coloneqq \Punct\parr{\sk,c^*_0}\),
        set \(m_0 \coloneqq 0\) and \(m_1 \coloneqq \sk\pars{c^*_0}\),
        set \(c'_1 \coloneqq \Enc'_1\parr{\pk',0;r_0}\),
        set \(c \coloneqq \Enc\parr{\pk,\fbox{0};r_1}\),
        set \(\overline{c}^* \coloneqq \parr{\overline{c}^*_0, c'_1, c}\),
        output \(\parr{\overline{\pk},\overline{c}^*,\sk\pars{c^*_0}}\).
    \end{hybrids}
    First, note that the output distribution of \(\H_1\) is the same as in the original pKDM-pKL security game.
    The distribution of \(\H_2\) is distributionally equal to the one of \(\H_1\).
    The indistinguishability between \(\H_3\) and \(\H_2\) follows from the IND-pKDM security of \(\PKE'\).
    The indistinguishability between \(\H_4\) and \(\H_3\) follows from the IND-CPA security of \(\PKE\).
    Lastly, the distribution of \(\H_4\) is independent of the choice bit \(b\).
\end{proof}



\begin{lemma}
    Let \(\FHE \coloneqq \parr{\Gen, \Enc, \Dec, \Eval}\) be a IND-CPA secure FHE scheme for \(\NC^1\).
    Let \(\PKE' \coloneqq \parr{\Gen', \Enc'_0, \Enc'_1, \Dec', \Punct'}\) be a ppPKE.
    Then there exists a pKDM-pKL secure ppFHE \(\overline{\FHE} \coloneqq \parr{\overline{\Gen}, \overline{\Enc}_0, \overline{\Enc}_1, \overline{\Dec}, \overline{\Punct}, \overline{\Eval}}\) for \(\NC^1\).
\end{lemma}