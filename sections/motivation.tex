\section{Preliminaries}

\begin{frame}
    \frametitle{Verifiable Functions}
    \begin{block}{Verifiable Random Functions}
        \begin{itemize}
            \item \(\vrfGen\parr{1^\secpar} \mapsto \parr{\mathsf{vk}, \mathsf{sk}}\)
            \item \(\vrfEval\parr{\mathsf{sk}, x} \mapsto \parr{\gimage, \pi}\)
            \pause
            \item \(\vrfVer\parr{\mathsf{vk}, x, \gimage, \pi} \mapsto b \in \bit\)
        \end{itemize}
        \pause
        Guarantees:
        \begin{itemize}
            \item Pseudorandomness as for standard PRFs but with the \(\mathsf{vk}\) and \(\vrfEval\) queries
            \pause
            \item Unique Provability:
            \\
            For all possible \(\vk\) (not necessarily generated by \(\vrfGen\)),
            all inputs \(x\),
            all images \(\gimage_1,\gimage_2 \in \gG\) and
            all possible proofs \(\pi_1,\pi_2\) it holds that
            \[\vrfVer(\vk,x,\gimage_1,\pi_1) = 1 \land \vrfVer(\vk,x,\gimage_2,\pi_2) = 1 \implies \gimage_1 = \gimage_2\]
        \end{itemize}
    \end{block}
\end{frame}

\begin{frame}
    \frametitle{Motivation}
    \begin{block}{Selected VRF constructions}
        \begin{tabular}{l|c|c|c|c|c}
            \small
            Reference & degree & \(|\mathsf{vk}|\) & \(|\pi|\) & assumption & remark \\
            \hline
            \cite{C:Lysyanskaya02} & \(\lambda\) & \(2\lambda\) & \(\lambda\) & \(q\)-type & \\
            \cite{PKC:DodYam05} & \(1\) & \(2\) & \(1\) & \(q\)-type & small inputs \\
            \cite{TCC:HofJag16} & \(O(\lambda)\) & \(O(\lambda)\) & \(O(\lambda)\) & DLIN & \\
            \cite{PKC:Kohl19} & \(\proofsize\) & \(\mathrm{poly}(\lambda)\) & \(\proofsize\) & DLIN & \(\proofsize\in\omega(1)\) \\
          \end{tabular}
    \end{block}
    \pause
    \begin{mdframed}[backgroundcolor=black!10]
        Do standard assumptions yield VRFs with constant-size proofs?
    \end{mdframed}
    \pause
    \begin{block}{Notation}
        \begin{itemize}
            \item \(\bracket{\g} = \gG\) \hfill // source group
            \item \(\bracket{\gt} = \gT\) \hfill// target group
            \item \(\pair\parr{\g^a,\g^b} = \gt^{ab}\) \hfill// pairing operation
            \item \(\vk = \parr{\g^{v_1}, ..., \g^{v_\vksize}}\)
        \end{itemize}
    \end{block}
\end{frame}

\begin{frame}
    \frametitle{Model}
    \tikzmark{start}
    \begin{block}{Example~\cite{PKC:DodYam05}}
        \begin{itemize}
            \item \(\vk = \g^{v_1}\)
            \item \(\gimage_x = \gt^{1 / \parr{v_1 + x}}\)
            \item \(\gproof_x = \g^{1 / \parr{v_1 + x}}\)
            \item \(\vrfVer\parr{\vk, x, \gimage, \gproof} = 1 \iff \smashedoverbrace{\pair\parr{\g, \gproof_x} = \gimage}{1 \cdot \frac{1}{v_1 + x} = \frac{1}{v_1 + x}} \wedge \smashedoverbrace{\pair\parr{\g^x \cdot \vk, \gproof} = \gt}{\parr{x + v_1} \cdot \frac{1}{v_1 + x} = 1}\)
            \item \(q\)-Diffie-Hellman inversion assumption:
            \\
            given \(\g, \g^{\alpha}, \g^{\alpha^2}, ..., \g^{\alpha^q}\) compute \(\g^{1/\alpha}\)
        \end{itemize}
    \end{block}
    \pause
    \begin{block}{Our restrictions}
        \begin{itemize}
            \item Verification can be described as a set of \enquote{pairing equations}
            \pause
            \item \(\implies\) Images have \enquote{rational} form with small degree:
            \\
            \(\gimage_x = \gt^{\numerator_x\parr{v_1, ..., v_n} / \denominator_x\parr{v_1, ..., v_n}}\)
            \item Reductions are algebraic/generic
        \end{itemize}
    \end{block}
    \tikzmark{end}
    \pause
    \begin{tikzpicture}[remember picture, overlay]
        \draw[blue,thick,->] ($(pic cs:end) + (8.7,2.62)$) -- ($(pic cs:start) + (6.5,-3.1)$);
        \draw[blue,thick,->] ($(pic cs:end) + (8.7,2.62)$) -- ($(pic cs:start) + (9.5,-3.1)$);
        \pause
        \draw[blue,thick,->] ($(pic cs:end) + (2,1.67)$) -- ($(pic cs:start) + (2.15,-1.85)$);
        \pause
        \draw[blue,thick,->] ($(pic cs:end) + (3.7,1.67)$) -- ($(pic cs:start) + (2.9,-1.85)$);
    \end{tikzpicture}
\end{frame}

\begin{frame}
    \frametitle{Contributions}
    \begin{mdframed}[backgroundcolor=black!10]
        Do standard assumptions yield VRFs with constant-size proofs?
    \end{mdframed}
    \pause
    \begin{block}{Contributions}
        \begin{enumerate}
            \item
            Verification\only<2->{\footnote{With a technical restrictions.}} by (quadratic) pairing equations
            \\\(\implies\) degree is at most exponential in proof size
            \pause
            \item
            \(\mathcal{O}\parr{\log\parr{\secpar}}\) proof size
            \\\(\implies\)
            polynomial degree
            \\\(\implies\)
            univariate polynomial-size assumption is insufficient
            \pause
            \item
            \(\mathcal{O}\parr{1}\) proof size
            \\\(\implies\)
            constant degree
            \\\(\implies\)
            small-size assumption is insufficient
        \end{enumerate}
    \end{block}
\end{frame}