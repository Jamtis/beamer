% !TEX root = ../paper.tex
\section{Technical Overview}
\label{sec:overview}

\begin{theorem}
    Assuming indistinguishability obfuscation,
    puncturable PRF,
    statistically binding commitment,
    and pseudorandom generator,
    there exists a deterministic statistically sound NIWI in the CRS model.
\end{theorem}

\begin{proof}
    Let \(\iO\) be an indistinguishability obfuscator,
    let \(\PPRF = \parr{\KeyGen, \Eval, \Punct}\) be a 2-puncturable,
    let \(\parr{\Com,\Open}\) be a statistically binding commitment scheme,
    and let \(\PRG\) be a pseudorandom generator.
    \\
    We define the following \(\NIWI \coloneqq \parr{\Setup, \Prove, \Vfy}\):
    \begin{sitemize}
        \item \(\Setup\parr{1^\secpar}\):
        Sample \(k_1,k_2 \gets \KeyGen\parr{1^\secpar}\).
        \\
        Let \(\mathsf{P} = \iO\parr{\parr{x, w} \mapsto \parr{\pi_0,\pi_1}}\) where \(\pi_0 \coloneqq \PRG\parr{\Eval\parr{k_1,x||w}}\) and \(\pi_1 \coloneqq \Com\parr{w; \Eval\parr{k_2,\pi_0}}\).
        \\
        Let \(\mathsf{V} = \iO\parr{\parr{x, \pi} \mapsto R\parr{x,w} = 1 \wedge \pi_0 = \PRG\parr{\Eval\parr{k_1,x||w}}}\) where \(w \coloneqq \Open\parr{\pi_1, \Eval\parr{k_2,\pi_0}}\).

        \item \(\Prove\parr{\CRS,x,w} \coloneqq \mathsf{P}\parr{x,w}\)

        \item \(\Vfy\parr{\CRS,x,\pi} \coloneqq \mathsf{V}\parr{x,\pi}\)
    \end{sitemize}

    Correctness directly follows from the correctness of the underlying primitives.
    Since the verification algorithm checks \(R\parr{x,w}\) the NIWI is perfectly (statistically) sound if the obfuscator is perfectly (statistically) correct.
    Now, we show the witness indistinguishability of \(\NIWI\) by a sequence of hybrid games.
    For simplicity we assume a perfectly binding commitment scheme.%
    \footnote{If the commitment scheme is statistically binding, then the saem argumentation applies but one has to consider the event that the scheme is not perfectly binding.}
    Let \(L \in \NP\) be language with relation \(R\).
    Let \(\hat{x} \in L\) be a statement with witnesses \(\hat{w}^0,\hat{w}^1\) s.t.\ \(\parr{\hat{x},\hat{w}^0},\parr{\hat{x},\hat{w}^1} \in R\).
    Let
    \begin{bralign}
        \mathsf{P}_{\hat{\pi}^0,\hat{\pi}^1,k_1,k_2}
        \coloneqq
        \iO\parr{\parr{x,w} \mapsto \begin{cases}
            \hat{\pi}^0 & x||w = \hat{x}||\hat{w}^0
            \\
            \hat{\pi}^1 & x||w = \hat{x}||\hat{w}^1
            \\
            \pi_0 \coloneqq \PRG\parr{\Eval\parr{k_1,x||w}}, \pi_1 \coloneqq \Com\parr{w; \Eval\parr{k_2, \pi_0}} & \text{else}\end{cases}
        }
    \end{bralign}
    and let
    \begin{bralign}
        \mathsf{V}_{\hat{\pi}^0,\hat{\pi}^1,k_1,k_2}
        \coloneqq
        \iO\parr{\parr{x,\pi} \mapsto \begin{cases}
            1 & \pi = \hat{\pi}^0
            \\
            1 & \pi = \hat{\pi}^1
            \\
            R\parr{x,w} = 1 \wedge \pi_0 = \PRG\parr{\Eval\parr{k'_1,x||w}} & \text{else}
        \end{cases}}
    \end{bralign}
    where \(w \coloneqq \Open\parr{\pi_1, \Eval\parr{k_2,\pi_0}}\).
    \begin{hybrids}
        \item Sample \(\CRS \gets \Setup\parr{1^\secpar}\).
        Let \(\pi^* \coloneqq \mathsf{P}\parr{\hat{x}||\hat{w}^0}\).
        Output \(\parr{\CRS, \pi^*}\).
        Note that this hybrid corresponds to the honest proving only uses the witness \(\hat{w}^0\).

        \item Let \(k'_1 \coloneqq \Punct\parr{k_1, \hat{x}||\hat{w}^0, \hat{x}||\hat{w}^1}\),
        let \(r_0^b \coloneqq \Eval\parr{k_1,\hat{x}||\hat{w}^b}\),
        let \(\hat{\pi}_0^b \coloneqq \PRG\parr{r_0^b}\),
        let \(r_1^b \coloneqq \Eval\parr{k_2,\hat{\pi}_0^b}\),
        and let \(\hat{\pi}_1^b \coloneqq \Com\parr{\hat{w}^b; r_1^b}\) for both \(b \in \bit\).
        Let \(\CRS \coloneqq \parr{\mathsf{P}_{\hat{\pi}^0,\hat{\pi}^1,k'_1,k_2}, \mathsf{V}_{\hat{\pi}^0,\hat{\pi}^1,k'_1,k_2}}\).

        \item Sample \(r_0^b \gets \bit^\secpar\).

        \item Sample \(\hat{\pi}_0^b \gets \bit^{2\secpar}\).
        Denote by \(E\) the event that \(\hat{\pi}_0^0, \hat{\pi}_0^1 \not\in \PRG\parr{\bit^\secpar}\).

        \item Let \(k'_2 \coloneqq \Punct\parr{k_2, \hat{\pi}_0^0, \hat{\pi}_0^1}\).
        Let \(\CRS \coloneqq \parr{\mathsf{P}_{\hat{\pi}^0,\hat{\pi}^1,k'_1,k'_2}, \mathsf{V}_{\hat{\pi}^0,\hat{\pi}^1,k'_1,k'_2}}\).

        \item Sample \(r_1^b \gets \bit^\secpar\).

        \item Let \(\hat{\pi}_1^b \coloneqq \Com\parr{\hat{w}^{1-b}; r_1^b}\).
        Let \(\pi^* \coloneqq \mathsf{P}\parr{\hat{x}||\hat{w}^1}\).

        \item Sample \(r_1^b \coloneqq \Eval\parr{k_2,\hat{\pi}_0^b}\).

        \item Let \(\CRS \coloneqq \parr{\mathsf{P}_{\hat{\pi}^0,\hat{\pi}^1,k'_1,k_2}, \mathsf{V}_{\hat{\pi}^0,\hat{\pi}^1,k'_1,k_2}}\).

        \item Let \(\hat{\pi}_0^b \coloneqq \PRG\parr{r_0^b}\).
        
        \item Let \(r_0^b \coloneqq \Eval\parr{k_1,\hat{x}||\hat{w}^b}\).
        Let \(\pi^* \coloneqq \mathsf{P}\parr{\hat{x}||\hat{w}^1}\).
        Output \(\parr{\CRS, \pi^*}\).
        Note that this hybrid corresponds to the honest proving only uses the witness \(\hat{w}^1\).
    \end{hybrids}
    For any PPT distinguisher \(\D\) consider the probability \(\varepsilon_i \coloneqq \Pr[{\parr{\CRS,\pi^*} \gets \H_i}]{\D\parr{\CRS,\pi^*} = 0}\) that the distinguisher outputs \(0\) in hybrid \(i\).
    \\
    Because the circuits \(\mathsf{P}\) and \(\mathsf{P}_{\hat{\pi}^0,\hat{\pi}^1,k'_1,k_2}\), and \(\mathsf{V}\) and \(\mathsf{V}_{\hat{\pi}^0,\hat{\pi}^1,k'_1,k_2}\) are functionally equivalent it follows that \(\abs{\varepsilon_1 - \varepsilon_2} \leq \eta_{\iO}\parr{\secpar}\).
    Because \(\H_2\) and \(\H_3\) only use \(k_1\) to compute \(r_0^b\) it follows that \(\abs{\varepsilon_2 - \varepsilon_3} \leq \eta_{\PPRF}\parr{\secpar}\) by the pseudorandomness of the puncturable PRF.
    Because \(\H_3\) and \(\H_4\) use \(r_0^b\) only to compute \(\pi_0^b\) it follows that \(\abs{\varepsilon_4 - \varepsilon_4} \leq \eta_{\PRG}\parr{\secpar}\) by the pseudorandomness of the PRG.
    Note
    \begin{bralign}
        \varepsilon_4
        &=
        \underbrace{\Pr[{\parr{\CRS,\pi^*} \gets \H_4}]{\D\parr{\CRS,\pi^*} = 0 \given E}}_{\varepsilon'_4} \underbrace{\Pr[{\parr{\CRS,\pi^*} \gets \H_4}]{E}}_{\leq 1}
        \\
        &+
        \underbrace{\Pr[{\parr{\CRS,\pi^*} \gets \H_4}]{\D\parr{\CRS,\pi^*} = 0 \given \text{not } E}}_{\leq 1} \underbrace{\Pr[{\parr{\CRS,\pi^*} \gets \H_4}]{\text{not } E}}_{\leq 2^{1-\secpar}}
        \\
        &\leq
        2^{1-\secpar}
        +
        \varepsilon'_4
    \end{bralign}
    and
    \begin{bralign}
        \varepsilon_4
        &=
        \varepsilon'_4 \underbrace{\Pr[{\parr{\CRS,\pi^*} \gets \H_4}]{E}}_{\geq 1-2^{1-\secpar}}
        +
        \Pr[{\parr{\CRS,\pi^*} \gets \H_4}]{\D\parr{\CRS,\pi^*} = 0 \given \text{not } E} \underbrace{\Pr[{\parr{\CRS,\pi^*} \gets \H_4}]{\text{not } E}}_{\geq 0}
        \\
        &\geq
        \varepsilon'_4 \parr{1-2^{1-\secpar}}
        \ ,
    \end{bralign}
    hence \(\abs{\varepsilon_4 - \varepsilon'_4} \leq 2^{1-\secpar}\).
    The event \(E\) implies that the circuits \(\mathsf{P}_{\hat{\pi}^0,\hat{\pi}^1,k'_1,k_2}\) and \(\mathsf{P}_{\hat{\pi}^0,\hat{\pi}^1,k'_1,k'_2}\) are functionally equivalent because \(\Eval\) only gets called on PRG images and \(k'_2\) in punctured at a non-PRG image.
    Moreover, \(E\) implies that the circuits \(\mathsf{V}_{\hat{\pi}^0,\hat{\pi}^1,k'_1,k_2}\) and \(\mathsf{V}_{\hat{\pi}^0,\hat{\pi}^1,k'_1,k'_2}\) are functionally equivalent.
    Specifically, on input \(\hat{x}||\hat{\pi}^b\) both circuits outputs \(1\).
    On the other hand for any input \(\hat{x}||\widetilde{\pi}_0^b||\hat{\pi}_1^b\) with \(\widetilde{\pi}_0^b \neq \hat{\pi}_0^b\) the outputs is \(0\) because the recovered \(w \coloneqq \Open\parr{\hat{\pi}_1^b, \Eval\parr{k'_2,\widetilde{\pi}_0^b}} = \bot\) (this follows from the prefect binding property of the commitment) will never satisfy the relation \(R\).
    Consequently, \(\abs{\varepsilon'_4 - \varepsilon'_5} \leq \eta_{\iO}\parr{\secpar}\) by the indistinguishability of the obfuscations.
    \\
    In \(\H_5\) the key \(k_2\) is only used to compute \(r_1^b\) and derive the key \(k'_2\) it follows \(\abs{\varepsilon'_5 - \varepsilon'_6} \leq \eta_{\PPRF}\parr{\secpar}\) by the pseudorandomness of the puncturable PRF.
\end{proof}